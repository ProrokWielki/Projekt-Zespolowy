\documentclass[a4paper, 11pt]{article}
\usepackage{polski}
\usepackage[utf8]{inputenc}
\usepackage{latexsym}
\usepackage[english, polish]{babel}
\usepackage{indentfirst}
\usepackage{geometry}
\author{Paweł Warzecha \and Adam Burdykiewicz}
\title{Opis Projektu}
\frenchspacing

\begin{document}

% TODO
	% rozdzial 5
	% wysrodkowac tabelki
	% lepiej podpis nad czy pod tabelka?
	% Gantt

\maketitle

\newpage
\tableofcontents

\newpage

\section{Opis ogólny}
Przedmiotem projektu jest budowa działa laserowego wyposażonego w system wizyjny umożliwiający identyfikację celów. Zostanie użyty laser niewielkiej mocy (5000mW), wystarczającej jednak by przebić czarne balony, które zostaną zdefiniowane jako jego cel.

Tego typu urządzenia są budowane do eksterminowania komarów w Afryce, co ma zredukować ilość zachorowań na malarię. Taki system ma jednak znacznie większy potencjał, dlatego nasz projekt zostanie zbudowany w oparciu o oprogramowanie Open Source oraz będzie rozpowszechniany na jej zasadach. 
Umożliwi to swobodne dokonywanie modyfikacji w celu dopasowania narzędzia do zaistniałych potrzeb, amerykańska armia jest w trakcie opracowywania potężnej broni laserowej montowanej na pojazdach, potrafiącej samodzielnie eliminować wymierzone w nie rakiety, nim te zdołają wyrządzić szkody. Przeznaczyli na ten cel 25 milionów dolarów.\footnote{http://www.lockheedmartin.com/us/news/press-releases/2014/april/140424-mst-lm-wins-contract-to-develop-weapons-grade-fiber-laser.html} Jest to efekt wieloletniego, postępującego zwiększania efektywności i mocy laserów w granicach bezpiecznych poziomów emisji ciepła\footnote{http://hedpschool.lle.rochester.edu/2011SummerSchool/lectures/Patel.pdf (7. slajd)}

Spodziewanym efektem prac będzie działo laserowe zdolne do eliminowania celów w czasie rzeczywistym z dwoma trybami pracy, automatycznym i ręcznym. 
Sterowanie ręczne będzie się odbywało przy pomocy kontrolera XBOX 360 przewodowo podłączonego do komputera zewnętrznego, który będzie komunikował się z komputerem pokładowym przez sieć Wi-Fi. Tryb automatyczny będzie samodzielnie wybierał cele bazując na odczycie z kamery Kinect, dostarczającej rozmieszczenie obiektów w przestrzeni. Analiza i interpretacja tego obrazu umożliwi przyjęcie odpowiedniej pozycji przez serwomechanizmy sterujące laserem i oddanie strzału.

Całość zostanie zrealizowana na systemie złożonym z mikrokontrolera Raspberry Pi i laptopa Dell Inspiron 14 3451. Mikrokontroler będzie odpowiedzialny za przetwarzanie danych z kamery i konfigurowanie serwomechanizmów, natomiast na komputerze zewnętrznym będzie wykonywane odczytywanie danych z kontrolera i ich konwersja na konieczne do przyjęcia konfiguracje. Laptop umożliwi też operatorowi wzgląd w status działa i przerwanie oraz wznowienie jego pracy w razie potrzeby.

Poza zaplanowaniem i oprogramowaniem funkcjonalności każdego z komponentów, prace nad projektem mają również za zadanie opracowanie metod komunikacji między tymi dwoma systemami opartymi na różnej architekturze procesora.

Wyniki prac będą umieszczane na stronie www pod adresem \\ http://89.72.70.224/laser.html.

Przewidywany czas ukończenia projektu to 13 tygodni, rozpoczynając 7.03.2016, a kończąc 6.06.2016.

\section{Plan prac i rozkład obowiązków}

% Slabo to wyglada...
% Lepiej pisac dalsze akapity a wstawki na koncu wskocza?

\begin{table}[h]
\caption{Podział prac}
\label{Plan}

\begin{tabular}{|c|l|l|} \hline
Nr zadania & Osoba odpowiedzialna & Opis zadania \\ \hline
Z1 & Paweł Warzecha & Zarządzanie projektem \\ \hline
Z2 & Paweł Warzecha & Sprecyzowanie założeń projektowych \\
 & & Specyfikacja systemu \\
 & & Określenie kryteriów ewaluacji \\ \hline
Z3 & wszyscy & Zapoznanie się z narzedziami \\ \hline \hline
Z4 & Piotr Bachry & Zaprojektowanie architektury oprogramowania \\
 & & Sporządzenie dokumentacji \\ \hline
Z5 & Adam Burdykiewicz & Zaprojektowanie układu mechanicznego \\
 & & Sporządzenie dokumentacji \\ \hline
Z6 & Jacek Kamieniecki & Zaprojektowanie układu elektrycznego \\ 
 & & Sporządzenie dokumentacji \\ \hline \hline
Z7 & wszyscy & Zakupy - pozyskanie części \\ \hline
Z8 & Adam Burdykiewicz & Wykonanie układu mechanicznego \\
 & & Sporządzenie dokumentacji \\ \hline
Z9 & Kamil Orłow & Wykonanie układu elektrycznego \\
 & & Sporządzenie dokumentacji \\ \hline \hline
Z10 & Kacper Pawlak  & Utworzenie oprogramowania odpowiedzialnego \\
 & Bartosz Kowalski & za obsługę kamery \\
 & & Sporządzenie dokumentacji \\ \hline 
Z11 & Adam Burdykiewicz & Utworzenie oprogramowania odpowiedzialnego \\
 & & za obsługę kontrolera \\
 & & Sporządzenie dokumentacji \\ \hline
Z12 & Jacek Kamieniecki & Utworzenie oprogramowania odpowiedzialnego \\
 & & za obsługę serwomechanizmów \\
 & & Sporządzenie dokumentacji \\ \hline
Z13 & Kacper Pawlak  & Utworzenie oprogramowania odpowiedzialnego \\ 
 & Bartosz Kowalski & za przetwarzanie obrazów \\
 & & Sporządzenie dokumentacji \\ \hline \hline
Z14 & Piotr Bachry & Zintegrowanie wszystkich elementów \\
 & & skomunikowanie ich ze sobą \\
 & & Sporządzenie dokumentacji \\ \hline
Z15 & Kamil Orłow & Utworzenie interfejsu użytkownika \\
 & & Sporządzenie dokumentacji \\ \hline
Z16 & wszyscy & Testy i usuwanie błędów \\ \hline
Z17 & wszyscy & Ewaluacja \\ \hline
Z18 & Paweł Warzecha & Zebranie, zredagowanie i podsumowanie \\
 & & dokumentacji z wszystkich kroków \\ 
 & & Sporządzenie raportu końcowego \\ \hline
\end{tabular}
\end{table}

\newpage
\restoregeometry

Kamienie milowe:

\begin{itemize}
\item Projekty wszystkich układów (mechanicznego, elektrycznego) i architektury oprogramowania (Z2-Z6) \\
Termin: 12.04.2016
\item Fizycznie wytworzone komponenty (Z7-Z9) \\
Termin: 3.05.2016
\item Oprogramowanie związane z poszczególnymi komponentami (Z10-Z13) \\
Termin: 24.05.2016
\item Raport końcowy (Z1-Z18) \\
Termin: 6.06.2016
\end{itemize}

Tu będzie Gantt
\begin{figure}[h]
\caption{Wykres Gantta}
\label{Gantt}
\end{figure}


\newpage
\section{Doręczenia}

\begin{table}[h]
\caption{Doręczenia}
\label{Doręczenia}

\begin{tabular}{|l|l|l|} \hline
Data & Nawa & Postać \\ \hline
12.04.2016 & Projekty wszystkich układów & Oprogramowanie \\
 & (mechanicznego, elektrycznego) & Dokumentacja \\
 & architektura oprogramowania & \\ \hline

2.05.2016 & Fizycznie wytworzone komponenty & Oprogramowanie \\
 & & Dokumentacja \\ \hline
 
23.05.2016 & Oprogramowanie związane & Sprzęt \\
 & z poszczególnymi komponentami & Oprogramowanie \\
 & & Dokumentacja \\ \hline
6.06.2015 & Raport końcowy & Raport końcowy \\ \hline
\end{tabular}
\end{table}

\section{Budżet}

\begin{table}[h]
\caption{Potrzeby i koszty}
\label{Budżet}

\begin{tabular}{|l|r|} \hline
Potrzeba & Koszt \\ \hline
Laser 5000mW & 60,00 zł \\ \hline
2x Serwomechanizm & 20,00 zł \\ \hline
Plexa & 20,00 zł \\ \hline
Tranzystor MOSFET & 2,00 zł \\ \hline
Balony & 10,00 zł \\ \hline \hline
Łącznie & 112,00 zł \\ \hline
\end{tabular}
\end{table}

\newpage
\section{Zarządzanie projektem}


\newpage
\section{Zespół}
\begin{itemize}
\item Paweł Warzecha - koordynacja projektu 
\item Kacper Pawlak - obsługa kamery i algorytm przetwarzania obrazów 
\item Bartosz Kowalski - obsługa kamery i algorytm przetwarzania obrazów
\item Jacek Kamieniecki - projekt układu elektrycznego i algorytm sterowania serwomechanizmami
\item Adam Burdykiewicz - projekt i wykonanie układu mechanicznego, obsługa kontrolera
\item Kamil Orłow - wykonanie układu elektrycznego, interfejs użytkownika
\item Piotr Bachry - projekt architektury oprogramowania i skomunikowanie komponentów
\end{itemize}

% A TU NASZE ZDJĘCIE 8-]

\end{document}